\documentclass[a4paper,twoside, 11pt]{book}
\usepackage[utf8x]{inputenc}
\usepackage[T1]{fontenc}
\usepackage[ngerman]{babel}
\usepackage{amsmath}
\usepackage{amsfonts}
\usepackage{amsthm}
\usepackage{amssymb}
\usepackage{makeidx}
\usepackage{graphicx}
\usepackage{caption}
\usepackage{tabularx}
\usepackage[hidelinks]{hyperref}
\usepackage[numbered,framed]{matlab-prettifier}
\usepackage{csquotes}
\usepackage{listings}
\usepackage{xcolor}
\usepackage{floatrow}
\usepackage{siunitx}
\usepackage{ragged2e}
\usepackage{booktabs}
\usepackage[list=true, font=small, labelfont=bf, 
labelformat=brace, position=top]{subcaption}
\newcolumntype{Y}{>{\RaggedRight\arraybackslash}X} 
\author{Jonas Schrage}
\title{Numerical experiments in the determination of regression ellipses}
\date{}

\lstset{literate=%
	{Ö}{{\"O}}1
	{Ä}{{\"A}}1
	{Ü}{{\"U}}1
	{ß}{{\ss}}2
	{ü}{{\"u}}1
	{ä}{{\"a}}1
	{ö}{{\"o}}1
}
\captionsetup{
	justification = centering
}


\newcommand{\norm}[1]{\left\lVert#1\right\rVert}
\newcommand{\N}{{\mathbb N}}
\newcommand{\R}{{\mathbb R}}
\newcommand{\C}{{\mathbb C}}
\newcommand{\Z}{{\mathbb Z}}
\newcommand{\Q}{{\mathbb Q}}
\newcommand{\K}{\text{K}}
\newcommand{\coloneqq}{:=}
\newcommand{\hspf}{\hspace*{\fill}}
\newcommand{\Rd}{{\R^\text{d}}}
\newcommand{\Rdd}{{\R^{\text{d} \times }}}
\newcommand{\Diff}{\text{Diff}}
\newcommand{\totdeg}{\text{deg}_{tot}}


\definecolor{Darkgreen}{rgb}{0.13, 0.55, 0.13}
\definecolor{Blue}{rgb}{0.0, 0.28, 0.67}
\definecolor{Purple}{rgb}{0.0, 0.28, 0.67}
\definecolor{Orange}{rgb}{1.00,0.67,0.00}
\definecolor{code}{rgb}{0.0, 0.0, 0.0}
\definecolor{Decorators}{rgb}{0.5,0.5,0.5}
\definecolor{Numbers}{rgb}{0.5,0,0}
\definecolor{MatchingBrackets}{rgb}{0.25,0.5,0.5}
\definecolor{Keywords}{rgb}{0,0.60,0.30}
\definecolor{self}{rgb}{0,0,0}
\definecolor{Strings}{rgb}{0.6,0.63,0}
\definecolor{Comments}{rgb}{0,0.63,1}
\definecolor{Backquotes}{rgb}{0,0,0}
\definecolor{Classname}{rgb}{0,0,0}
\definecolor{FunctionName}{rgb}{0,0,0}
\definecolor{Operators}{rgb}{0,0,0}
\definecolor{Background}{rgb}{0.98,0.98,0.98}
\definecolor{colKeys}{RGB}{0,0,255}       % blau
\definecolor{colIdentifier}{RGB}{0,0,0}   % schwarz
\definecolor{colComments}{RGB}{34,139,34} % gruen
\definecolor{colString}{RGB}{160,32,240}  % violett

\captionsetup[subfigure]{labelformat=brace, subrefformat=brace}
\renewcommand{\thesubfigure}{\arabic{subfigure}}

\lstset{%
	language=Matlab,%
	basicstyle={\footnotesize\ttfamily},%
	breakautoindent=true,%
	breakindent=10pt,%
	breaklines=true,%
	captionpos=t,%
	columns=fixed,%
	commentstyle={\itshape\color{Darkgreen}},%
	extendedchars=true,%
	frame=single,%
	framerule=1pt,%
	identifierstyle={\color{colIdentifier}},%
	keywordstyle={\color{colKeys}},%
	keywordstyle=[2]\color{Orange},
	keywordstyle=[3]\color{Blue}\underbar,%
	morekeywords={%
		arguments,switch,case,otherwise,assert,clearvars,vpa,simplify,sym,scatter3%
	},%
	morekeywords=[2]{mustBeReal,mustBeMember,mustBeNumericOrLogical,mustBeText,mustBeInteger,mustBePositive},% argumentchecks
	morekeywords=[3]{},%custom functions
	numbers=left,%
	numbersep=1em,%
	numberstyle={\tiny\ttfamily},%
	showspaces=false,%
	showstringspaces=false,%
	stringstyle={\color{colString}},%
	morestring=[d]{"},%
	tabsize=4,%
	xleftmargin=1em,%
	xrightmargin=1em%
}


\newtheoremstyle{custom}% name
{\topsep}%      Space above, empty = `usual value'
{}%      Space below
{}% Body font
{}%         Indent amount (empty = no indent, \parindent = para indent)
{\bfseries}% Thm head font
{:}%        Punctuation after thm head
{4pt}% Space after thm head: \newline = linebreak
{}%         Thm head spec

\renewcommand{\thesubfigure}{\roman{subfigure}}
\renewcommand{\thechapter}{\arabic{chapter}}
\renewcommand{\thesection}{\arabic{chapter}.\arabic{section}}

\swapnumbers
\theoremstyle{custom}
\newtheorem{theorem}{Theorem}[subsection]
\renewcommand{\thetheorem}{\arabic{chapter}.\arabic{section}.\arabic{theorem}}
\newtheorem{proposition}[theorem]{Satz}
\newtheorem{lemma}[theorem]{Lemma}
\newtheorem{cor}[theorem]{Korollar}


\theoremstyle{custom}
\newtheorem{definition}[theorem]{Definition}
\newtheorem{conj}[theorem]{Vermutung}
\newtheorem{example}[theorem]{Beispiel}
\newtheorem{examples}[theorem]{Beispiele}

\newtheorem{remark}[theorem]{Bemerkung}
\newtheorem{note}[theorem]{Notiz}


\begin{document}
	\bibliographystyle{ieeetr}
	\maketitle
	\newpage
	\tableofcontents
	\newpage
	\chapter{Equations of higher degree}
	\section{3 Gleichungen 3. Grades in 3 Variablen}
	\newpage
	\section{Aufteilen der Variablen}
	\begin{proposition}\label{prop:gen_inv}
		Für eine Matrix $A \in K^{m\times n}$ gilt:
		\begin{itemize}
			 \item[$-$] $A$ besitzt eine linke Inverse genau dann, wenn $Rang(A) = n$
			 \item[$-$] $A$ besitzt eine rechte Inverse genau dann, wenn $Rang(A) = m$
		\end{itemize}
		Insbesondere existiert im Falle $n<m$ keine linke Inverse und im Falle $n>m$ existiert keine rechte Inverse.
	\end{proposition}

	\begin{proof}
		content
	\end{proof}

	Analog zu dem Vorgehen in \ref{} lassen wir $x$ erstmal außen vor und teilen die verbleibenden Kombinationen der Unbestimmten $Y,\ Z$ in zwei Gruppen auf.
	Dies führt zu den 10 Kandidaten 
	\begin{equation}
		\{Y^3,Z^3,Y^2Z,YZ^2,Y^2,Z^2,YZ,Y,Z,1\},
	\end{equation}
	welche wir nun in die zwei Gruppen 
	\begin{align}
		G1 &\coloneqq \{\mu_1,\ldots,\mu_m\} \ \text{und} \\
		G2 &\coloneqq \{\eta_1,\ldots,\eta_n\}
	\end{align}
	aufteilen wollen.
	
	Für $m$ Elemente in der ersten Gruppe und $n \coloneqq 10-m$ ergibt sich das Gleichungssystem in Matrixform als
	\begin{equation}
		-\underset{\scriptscriptstyle 3\times m}{A} \cdot \left(\mu_1,\ldots,\mu_m\right)^{\intercal} = \underset{\scriptscriptstyle 3\times n}{P} \cdot \left(\eta_1,\ldots,\eta_n\right)^{\intercal}
	\end{equation}
	Im nächsten Schritt isolieren wir die $\mu_k$'s indem wir beide Seiten mit der linksinversen Matrix zu A multiplizieren. Laut Satz \ref{prop:gen_inv} existiert diese aber nur im Falle $m \leq 3$.
	
	Für $m \leq 3$ erhält man also 
	\begin{equation}\label{eqn:cub_inv_eqs}
		\left(\mu_1,\ldots,\mu_m\right)^{\intercal} = -\underset{\scriptscriptstyle m\times 3}{A^{-1}}\cdot \underset{\scriptscriptstyle 3\times n}{P} \cdot \left(\eta_1,\ldots,\eta_n\right)^{\intercal}
	\end{equation}
	
	Um das Vorgehen weiter übertragen zu können müssen wir $m$ Gleichungen bei denen wir $\mu_1, \mu_2$ und $\mu_3$ verwenden um diese mithilfe von \eqref{eqn:cub_inv_eqs} zu substituieren.\\
	Damit die Substitution zur Elimination der Unbestimmten $\mu_1, \mu_2, \mu_3$ führt, darf es nach dem Einsetzen nur noch Terme in $\mu_1, \ldots, \mu_m$ und $\eta_1,\ldots,\eta_n$ geben. Es dürfen dabei keine Monome $M$ mit $\totdeg\left(M\right)\geq 4$ aus den Unbestimmten $Y,Z$ entstehen.
	\section{Substitution}
	\subsection{Fall 1 $m=3, n=7$}
	Die vier Monome $Y^{3}$, $Y^{2}Z$, $YZ^{2}$ und $Z^{3}$ haben alle einen totalen Grad von $3$.
	Da wir nur drei Monome in die Gruppe $G_1$ zuteilen, besitzt Gruppe $G_2$ mindestens eines dieser Monome.
	
	Die Gleichungen zur Substitution haben folgende Form
	
\begin{gather}
		\begin{alignedat}{1}
	Y^{a_1}Z^{\tilde{a}_1} \cdot \mu_1	&= \mu_2 \cdot Y^{b_2}Z^{\tilde{b}_2} \\
	Y^{a_2}Z^{\tilde{a}_2} \cdot \mu_2	&= \mu_3 \cdot Y^{b_3}Z^{\tilde{b}_3} \ ,\\
	Y^{a_3}Z^{\tilde{a}_3} \cdot \mu_3	&= \mu_1 \cdot Y^{b_1}Z^{\tilde{b}_1}
	\end{alignedat}
\end{gather}

wobei $a_k, \tilde{a}_k ,b_k , \tilde{b}_k \in \N_{0}$ sind.
 Die paarweise verschiedenen $\mu_k$ führen dazu, dass zwangsweise $a_k + \tilde{a}_k \geq 1$ oder $b_k + \tilde{b}_k \geq 1$ gilt.\\

Beim Substituieren nach der Vorschrift aus Gleichung \eqref{eqn:cub_inv_eqs} entsteht dann ein Monom $M$ mit $\totdeg\left(M\right) = 4$. Dies ist ein Monom welches nicht in unserem kubischen Polynom \ref{} vorkommt und somit ein Widerspruch zu \ref{}.

Mit analoger Begründung kann das Verfahren auch in den verbleibenden Fällen $m<3$ nicht funktionieren.

\section{Anpassung des Gleichungssystems}
Aus Symmetriegründen bietet sich bei den zehn Monomen an $n=m=5$ zu wählen.
Damit unsere Matrix $A$ aus \eqref{eqn:cub_inv_eqs} eine linke Inverse besitzt brauchen wir also mindestens $5$ Gleichungen.

Mit der Wahl von $G_{1} = \{Y^{3},Y^{2}Z,YZ,YZ^{2},Z^{3}\}$ und den Substitutionsgleichungen
\begin{gather}
	\begin{alignedat}{-1}
	Z \cdot& \left( Y^{3}\right)  &= \left( Y^{2}Z\right) \cdot& Y\\
		Z\cdot& \left(Y^{2}Z\right) &= \left(YZ^{2}\right) \cdot& Y\\
		Y\cdot&\left( Z^{3}\right) &= \left(YZ^{2}\right) \cdot& Z\\
		&\left(Y^{2}Z\right)  &= \left(YZ\right) \cdot& Y\\
		&\left(YZ^{2}\right)  &= \left(YZ\right) \cdot& Z\\
	\end{alignedat}
\end{gather}
ergeben sich nach Einsetzen von $G_{2} = \{Y^2,Z^2,Y,Z,1\}$ nur Monome von totalen Grad drei oder kleiner.

Substituiert man jetzt im resultierenden Gleichungssystem nur die Monome $Y^{3}$,$Y^{2}Z$,$YZ^{2}$,$Z^{3}$, so erhält man fünf quadratische Gleichungen in $X,Y,Z$. Dieses System können wir nun mit dem Verfahren \ref{} lösen.
	\newpage
	\pagestyle{plain}
	\nocite{*}
	\bibliography{bibliography.bib}
\end{document}